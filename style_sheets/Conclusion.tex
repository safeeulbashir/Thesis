\chapter{Conclusion}
An enormous amount of research has been performed to analyze the foraging strategies of ants. Scientists performed many biological experiments to discover what strategies ants follow to forage from the environment. \par
Although we have talked about the combination of three strategies for foraging, it was unclear which of the strategies ants follow for recruiting from large food sources. Several indoor experiments on harvester ants which used high definition cameras and chemical detectors have shown that they lay pheromone trails for other ants to follow. \par
As the amount of pheromone laid by the ants are very small, it was difficult to identify the existence of chemicals in the outdoor field experiments. So, we used this mathematical approach to detect the pheromone laying events of ants in the outdoor field experiments. \par
In simulations, we could detect change points in every single experiment.
Since we can detect change points in the majority of the field experiments, we infer that all three species of ants either use pheromone to recruit from large clumped food sources or exclusively uses sitefidelity which generates change points. \par
In \textit{memory only} simulation environment ants used sitefidelity extensively, which resulted in the drastic change in their foraging rate. As a result, we can see picks and hikes in their foraging rate which was detected by our change point detection algorithm. But when we have tuned the simulations to use both memory and communication, we observe ants did not use memory much, even if they use, it didn’t affect their foraging rate, so it is only detected for pheromone.\par

We have only used binary segmented cumulative sum methods on raw data for detecting the change points. In future, we can try to observe and find the pattern of changes in the foraging rate when the pheromone is laid, and use the statistical data from the simulation to infer the pheromone laying events in the field experiment data.
