\chapter{Introduction}
Social insects are  species that live in colonies and manifest three characteristics \cite{SocialInsect}; a. group integration\cite{anderson2001individual}, b. division of labor\cite{beshers2001models} and c. overlapping generations\cite{wilson2005eusociality}. These creatures have survived many mass level extinction events. Millions of years of genetic evolution helped them to adapt to the environment and to master the strategies of survival.  \par
Their strategies to avoid congestion and to optimize their movements to forage in most efficient ways without any central authority have attracted researchers and scientists\cite{narzt2010self}. In modern computer science, machine learning\cite{dorigo1997ant}, complex interactive networks\cite{he2011ant}, parallel computing\cite{bonabeau2000inspiration} and many other topics have been inspired by the studying and modeling of ants. \par
Harvester ants forage during the morning or in the evening sessions during the summer\cite{hobbs1985harvester, whitford1975factors}.  In this study, we will mostly talk about \textit{Progonomyrmex} species, which are group foragers\cite{whitford1978foraging}. Foraging activities of harvester ants depend on many factors such as temperature, light, and availability of seeds\cite{whitford1975factors}.\par 
Social insects such as ants sometimes use pheromone to communicate with each other to forage\cite{jackson2006communication}. The previous study has shown that they follow three strategies to forage\cite{flanagan2012quantifying,flanagan2011ants,letendre2013synergy}. Ants use memory to remember the location of the food source. They communicate with other ants using pheromone, and they perform random walk in search of food.\par 
Foraging depends mostly on how  food is distributed in the  environment\cite{traniello1989foraging,letendre2013synergy}. To analyze the foraging strategies of ants, field experiments have been conducted on three species of \textit{Pogonomyrmex} desert harvester ants. Foods were distributed around the nest in different distributions to observe the effect of food density on foraging. It was demonstrated that ants take some time to discover the large pile of seeds, but they start recruiting from the food source once they discover it\cite{flanagan2012quantifying}. 

Based on this behavior, an agent-based model called the Central Place Foraging Algorithm(CPFA)\cite{hecker2015beyond} was developed by Hecker and Moses. The purpose of CPFA is to collect resources from the spatial environment by using the strategies mentioned above: memory, communication and random walk\cite{collett2010desert,flanagan2012quantifying,letendre2013synergy}.However, it is not clear which strategies are used under which conditions in real ants\cite{tarasewich2002swarm}.  It is possible that when ants discover a pile they lay pheromone trail for other ants to follow. When other ants start following the pheromone trail, their foraging rate goes up for that pile. We use a change point detection algorithm to detect that change in foraging rate. Our goal is to detect when ants use pheromones rather than site fidelity. Because field data are noisy and we have no ground truth, we use simulations to test how reliably change point detection is at identifying pheromone use. We then use the method to identify change points and infer pheromone use on the field data.

We used the CPFA to simulate the field experiments. The environment of the CPFA has been designed to mimic the field experiments for three different species \textit{P. rugosus}, \textit{P. maricopa} and \textit{P. desertorum}. We have implemented different change point detection algorithms on the simulated data\cite{fryzlewicz2014wild, scott1974cluster, kukulski2000normal}. The change point detection algorithms were tuned to detect change points when the pheromone is laid. From the simulation, we know exactly when the pheromone was laid and site fidelity was used, thus we verified our change point detection algorithms by using the simulation data. Based on the results of the simulation we have selected best change point detection method and applied the best algorithm with tuned parameters on the field data. 

We observe that in simulation ants use pheromone more when the food sources are clumped, And they discover the pile more frequently if the food source is large. They don't use the pheromone to recruit from the food that is scattered in the environment. 


\clearpage
\section{\label{section:Contribution and Organization}Contribution and Organization}
Chapter 2 describes species of the ants that we have used in our experiments, the ant inspired model CPFA and the Genetic Algorithm for tuning the parameters.\\
\\
Chapter 3 describes the procedure that we have followed to investigate the problem. It includes the simulation environment, parameter settings for genetic algorithm, methods and efficiencies of different approaches of change point algorithms. \\
\\
Chapter 4 includes the results of different change point detection methods and the result of applying best change point detection algorithm on the field data. \\
\\
Chapter 5 includes conclusion and future works. \\
\\
Appendices are included at the end.
