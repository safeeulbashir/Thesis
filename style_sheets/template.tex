% Example template for using the unmeethesis style
% This example is for a Master's candidate in Mathematics
% It contains examples of front matter and most sections that the
% typical graduate student would need to include
% By: N. Doren 02/10/00
%     Minor mods by N. Doren 08/26/11

% Use the following specification for BOTTOM page numbering:
\documentclass[botnum, fleqn]{unmeethesis}
                 % OR
% Use the following specification for TOP page numbering:
% \documentclass[fleqn]{unmeethesis}
\usepackage{graphicx}
\usepackage{caption}
\usepackage{lscape}
\usepackage{rotating}
\usepackage{multirow}
\usepackage{textcomp}
%\usepackage[linesnumbered,algoruled,boxed,lined]{algorithm2e}
\usepackage[chapter]{algorithm}
\usepackage{subcaption}
\usepackage{algorithm}
\usepackage{algpseudocode}
\renewcommand{\algorithmicrequire}{\textbf{Input:}}
\renewcommand{\algorithmicensure}{\textbf{Output:}}
\usepackage{hyperref}
\hypersetup{
	colorlinks=true,
	linkcolor=black,
	filecolor=magenta,      
	urlcolor=blue,
}
\graphicspath{ {../Images/} }

\begin{document}

\frontmatter

% Uncomment the next command if you see weird paragraph spacing:
% That is, if you see paragraphs float with lots of white space
% in between them:

% \setlength{\parskip}{0.30cm}


\title{Detection of Pheromone Laying Event in Foraging Data of Harvester Ants Using Change Point Analysis Method}

\author{Safeeul Bashir Safee}

\degreesubject{M.S., Computer Science}

\degree{Master of Science \\ Computer Science}

\documenttype{Thesis}

\previousdegrees{B.Sc., Chittagong University of Engineering \& Technology, 2011}

\date{May, \thisyear}

\maketitle

%\makecopyright
%Copyright page is no longer necessary D. Murrell

\begin{dedication}
   To my parents and my elder brother for their support,
   encouragement and the Corvette they're giving me for graduation. \\[3ex]
\end{dedication}

\begin{acknowledgments}
   \vspace{1.1in}
  I would like to thank my thesis supervisor Professor Dr. Melanie E. Moses, for her enormous support for selecting the topic to study. Without her guidance, it would have been a tough journey for me in this field. I would also like to thank Dr. Tatiana P. Flanagan for her initial guidance to me for this topic. My heart full of gratitude to Professor Dr. Abdullah Mueen for his guidance towards my degree. 
  I am thankful to all the members of Moses Biological Computation Lab for their suggestions towards my research. 
\end{acknowledgments}

\maketitleabstract %(required even though there's no abstract title anymore)

\begin{abstract}
 Communication is an important factor in the foraging performance of
 social insects, like ants. During foraging, ants keep track of the food
 sources by using memory (site fidelity) or communicating it through pheromones. Previous field experiments showed that the rate of seed collection depends on the distribution of food in the environment. If food is spatially clustered, then ants recruit nest mates to collect from large clusters. However, we don't know when the recruitment occur. To explore this question, we analyzed foraging rates on different sizes of piles in the simulation.  Using a power law distribution to arrange seeds in piles of different sizes, we observed that for significantly large piles of seeds, the ants take more time to discover a pile, but once discovered, seeds are collected at an increased rate from that pile. We also observed that ants may repeatedly lose track of found piles and then re-find them. Using change point analysis on seed intake time series, we were able to trace the discovery of piles by detecting changes in the foraging rate.  We use simulations to determine how to correlate change points with recruitment events, and then use that relationship to infer when recruitment occurs in field data.
\clearpage %(required for 1-page abstract)
\end{abstract}
\tableofcontents
\listoffigures
\listoftables
\listofalgorithms
\addtocontents{loa}{\def\string\figurename{Algorithm}}
\begin{glossary}{Longest  string}
	\item[GA]
	Genetic Algorithm
	\item[CPFA]
	Central Place Foraging Algorithm
\end{glossary}
\mainmatter
\chapter{Introduction}
Social insects are basically species that lives in colonies and manifest three characteristics \cite{SocialInsect}. a. Group integration\cite{anderson2001individual}, b. Division of labor\cite{beshers2001models} and c. Overlapping of generations\cite{wilson2005eusociality}. These creatures have survived many mass level extinction events that happened in the earth. Millions of years of genetic evolution to survive in the hostile environment helped them to adapt to the environment and master the strategies of survival.  \par
Their strategies to avoid congestion and optimize their movements to move or forage in most efficient ways without any central authorities has attracted so many researchers and scientists over the centuries\cite{narzt2010self}. As a result, we got a new branch of science which is called myrmecology. In modern computer science, machine learning\cite{dorigo1997ant}, complex interactive networks\cite{he2011ant}, parallel computing\cite{bonabeau2000inspiration} and many other topics have been inspired by the studying and modeling of ants. \par
Harvester ants are social insects. They forage from the environment. Most of their foraging activities are during the morning or in the evening sessions. And their foraging activity is at the top during the summer\cite{hobbs1985harvester, whitford1975factors}.  In this study, we will mostly talk about \textit{Progonomyrmex} species which are group foragers\cite{whitford1978foraging}. Foraging activities of harvester ants including \textit{Progonomyrmex sp.} depends on many factors like environment temperature, light, and availability of seeds\cite{whitford1975factors}.\par 
Social insects like ants use pheromone to communicate with each other to perform their daily activities which also includes foraging\cite{jackson2006communication}.
  The previous study has shown that they follow three strategies to forage from the environment\cite{flanagan2012quantifying}. Ants use memory to remember the location of the food source. They communicate with other ants using pheromone and they perform the random walk in the spatial dimension in search of food.\par 
  Foraging of ants from a particular food source depends mostly on the how  food is distributed in the environment\cite{traniello1989foraging}. To analyze the foraging strategies of ants field experiments has been conducted on three species of Pogonomyrmex desert harvester ants. Foods were distributed among the nest in different distributions to observe the effect of food density on foraging. It was demonstrated that ants take some time to discover the large pile of seeds, but they start recruiting from the food source once they discover it\cite{flanagan2012quantifying}. 

Based on this behavior an agent-based model CPFA\cite{hecker2015beyond} is developed by Moses Biological Computation Lab. CPFA is implemented on various platforms. The purpose agents of CPFA is to collect resources from the spatial environment by the strategies mentioned above. 

As mentioned previously, they use three different strategies to forage\cite{collett2010desert,flanagan2012quantifying}, it was not clear what strategies they use to recruit from a clumped food source\cite{tarasewich2002swarm}. Our initial observation showed that when ants discover a pile they lay pheromone trail for other ants to follow. When other ants start following the pheromone trail, their foraging rate goes up for that pile. We have used change point detection algorithm to detect that change in foraging rate. Our goal was to determine what strategies ants mostly use to forage from clumped food source. It was difficult to determine from the field data whether the detected change points indicates the pheromone laying event. To do so we have used simulations. 

We have used CPFA to simulate the field experiments. The environment of CPFA has been tuned to mimic the field experiments for three different species \textit{P. Rugosus}, \textit{P. Maricopa} and \textit{P. Desertorum}. We have implemented different change point detection algorithms on the simulated data\cite{fryzlewicz2014wild, scott1974cluster, kukulski2000normal}. The change point detection algorithms were tuned to detect change points when the pheromone is laid. For each of the species the change point detection algorithm has different sets of parameters. As from the simulation, we know exactly when the pheromone was laid and site fidelity was used, we verified our change point detection algorithms by using the simulation data. Based on the results of the simulation we have selected best change point detection method. And applied the result to the field data. 

We observe ants use pheromone more when the food sources are clumped. And they discover the pile more frequently if the food source is large. They don't use the pheromone to recruit from the food source that is scattered in the environment. 

As we mentioned Foraging of ants depends on many factors including temperature, availability of food, distance from food to the nest and so many\cite{whitford1975factors,gordon1996founding}. So in some of the field experiments, ants did not collect enough seeds. Which is why we were unable to detect any change points in some of the experiments for each of the species. Since the CPFA does not depend on these conditions, we have detected change points in each of the simulations.
\clearpage
\section{\label{section:Contribution and Organization}Contribution and Organization}
Chapter 2 includes background study for solving this problem. It describes the ants that we have used in our experiments, the ant inspired model CPFA and the Genetic Algorithm for tuning the parameters.\par
Chapter 3 describes the procedure that we have followed to investigate the problem statement. It includes the simulation environment, parameter settings for genetic algorithm, methods and efficiencies of different approaches of change point algorithms. \par
Chapter 4 includes the results of different change point detection methods. And the result of best change point detection algorithm on the field data. \par
Chapter 5 includes conclusion and future works. 


\chapter{Background Study}
Ants are social insect. These small tiny creatures have survived major extinction level events in the world. They have been evolved millions of years to survive in the worlds. Their strategies to find resources for their survival are fascinating. We ran several experiments on desert harvester ants at the field to observe how they forage. We have selected three different species of harvester Ants to observe their foraging strategies. Those ants are \textit{P. Rugosus}, \textit{P. Desertorum} and \textit{P. Maricopa}. Our goal was to figure out how they find resources from an environment and what is the effect of the distribution of information on their foraging. So we ran experiments on ants. Seeds in the fields are distributed in a donut shape ring. The area of the food distribution is scaled with the colony size of ants. For example, Desertorum was the smallest in colony size (77$\pm$296), so the donut ring radius of food was 1.5 to 3 meters, Rugosus has a colony size of 1712$\pm$174. So the radius of food distribution for Rugosus was in 5-10 meters. The seeds are organized in a power law distribution around the nest.
\section{\label{section:Power Law}Power Law}
In power law distribution of food, seeds are distributed into multiple piles of different pile size. For example, for power rank 5 of power law distribution total number of seeds will be 1024. And it is divided into 4 types of 256 seeds in each type. One large pile of 256 seeds are placed all together in a certain position. Next 256 seeds are divided into 4 equal sizes of 64 seeds and placed around the nest. Next 256 seeds are equally divided into 16 piles of 16 seeds and placed around the nest inside the ring. Rest of the seeds are distributed uniformly around the nest.
\begin{figure}[h]
	\includegraphics[width=\textwidth]{DonutShape.png}
	\caption{Distribution of Seeds in the field experiment for power law distribution with power rank 5. Red Pile indicates one large pile of 256 seeds. 4 purple piles represent 4 large piles of 64 seeds, Green color represents 16 piles of 16 seeds and blue seeds are 256 random seeds}
\end{figure}
\section{\label{section:CPFA}CPFA}
After the distribution of foods across the nest we observe their collection of seeds. It is observed that ants need longer time to discover the large pile with 256 seeds then the piles with small amount of seeds. But once the seeds are discovered, they started recruiting from the piles. Studies showed that information is transferred among the ants for larger piles with more seeds. We have plotted the collection of seeds from the field experiments and discovered ants walk randomly to collect seeds. Once they get the seed they bring it back to the nest. But if they discover a large food source they share the information with others to recruit from the food source. Once the information is shared more ants come into recruitments and while collecting this food they also share information with other ants. As soon as ants start recruiting from the food source, their foraging rate goes up. But the food source starts losing the number of seeds. As the number of seeds starts decreasing from the food source the recruitment of amount of ants also starts decreasing. Based on this behavior, an algorithm was proposed to simulate the behavior of ants. It is called Central Place Foraging Algorithm(CPFA). CPFA is an agent-based model where agents are programmed to follow ant’s strategy to collect seeds. In CPFA Ants follows three methods to collect seeds. 

\chapter{Method}
\begin{figure}[h]
	\includegraphics[width=\textwidth]{Method.png}
	\caption{Steps of Change Point Analysis}
\end{figure}
\section{\label{section:Setting Simulation Environment}Setting Simulation Environment}
We tried to setup the simulation environment as like as the field experiment environment. So we distributed the resources in Donut Shapes. We had three different setups for three different species of ants. For \textit{Rugosus} the distribution of the resources was in power law and the power rank was $5$. Total $1024$ seeds are distributed in 4 different densities. $256$ seeds are piled up in one single pile. There were $4$ piles of $64$ seeds in each. $16$ piles had $16$ seeds each and rest of the $256$ seeds were distributed uniformly among the nest in the donut shape ring. Although in the field experiment, the area of the donut shape was proportional to the colony size. In here we tried to keep the area of the donut shape ring constant. The inner radius of the ring was $5$ meter and the outside radius was $10$ meter. The total duration of each experiment was $90$ minutes (We collected data from field experiments for $90$ minutes only). The total arena size was $20\times20$ meter. We kept the arena into this size and bounded the agents to search in this arena.  The setup is varied for \textit{Maricopa} and \textit{Desertorum}. The following table represents the environmental setup of simulations for \textit{P. Rugosus}, \textit{P. Maricopa} and \textit{P. Desertorum}.
\begin{table}
	\begin{tabular}{ |p{0.3\textwidth}|p{0.3\textwidth}|p{0.3\textwidth}| } 
		\hline
		\textbf{Species} & \textbf{Number of Seeds} & \textbf{Radius of Seed Distribution} \\
		\hline 
		\textit{P. Rugosus} & 1024 & 5-10 meter\\ 
		\hline
		\textit{P. Maricopa} & 128 & 1-3 meter\\ 
		\hline
		 \textit{P. Desertorum} & 128 & 1-3 meter\\
		\hline
	\end{tabular}
	\caption{Environmental Setup of simulation for three species}
\end{table}
\begin{figure}[!h]
	\frame{\includegraphics[width=\textwidth]{ArGOS.jpg}}
	\caption{Example setup of a simulation environment for \textit{P. Rugosus} with total 1024 seeds. The setup showing three different types of piles. One large pile of $256$ seeds, Four piles of $64$ seeds and sixteen piles of $16$ seeds. $256$ random seeds are distributed uniformly inside the ring. }
\end{figure}
\section{\label{section:Tuning Parameters using Genetic Algorithm}Tuning Parameters using Genetic Algorithm}
   To analyze the data, we have tuned the parameters of CPFA. As stated above in the background study, enormous amount of parameters for CPFA can be used to evaluate the fitness. But we have used the genetic algorithm to achieve the optimum set of parameters. We have divided the simulations into three categories to tune the GA for three different environments. 
   \begin{enumerate}
   	\item \textbf{Pheromone Only Parameters:}  For this type we have restricted the use of site fidelity. Which means that the probability of using site fidelity is $20$. Rest of the parameters evolved using the GA. 
   	\item \textbf{Site fidelity Only Parameters:} In this types of experiment we restricted the use of pheromone. For this case ants can only use site fidelity and random walk to collect resources
   	\item \textbf{Using Both site fidelity and pheromone:} This environment represents the actual field experiment condition. Here agents use both site fidelity and pheromone along with the random walk.    	
   \end{enumerate}
For each type of environment, we have tuned the parameters to obtain maximum fitness using Genetic Algorithm. Initially, we have created the one hundred set of parameters. And then evaluated their fitness. The best genome is selected for crossover and mutation for next generation. We continued this process until we obtain the best fitness genome or parameter set. The GA is terminated either when the parameters are converged, or it reaches to generation $50$.  
For each parameter set in a population, fitness is tested for four different random seeds. The Random seed is the variable which controls the variables of a simulation. After evaluation of each random seed for one parameter set, we have calculated the average seed collection to define the fitness of that particular set of parameters. These random seeds are basically numbers which are fixed for each generation. For each generation, we have selected four different numbers for the random seeds and then evaluated all the parameters for those values.
For GA we can confine particular parameters to a certain range of values by setting this in evolver.cpp file. For Example, for "Pheromone Only" Environment we restricted the values of site fidelity to $(20,20)$ where each value inside the parenthesis represents upper bound and lower bound.
To calculate the fitness for each parameter set it takes evaluating the fitness function for four times due to four different random seeds, which means for each generation it needs evaluating the objective function for $400$ times. So over $50$ generations, it will need $2000$ evaluations of the objective function. This can take a lot of time if we perform the evaluation sequentially. To remove this bottleneck, we have used multi-threading of genetic algorithm by evaluating multiple objective functions simultaneously. We have used GA Lib Genetic Algorithm Package. The software for this work used the GAlib genetic algorithm package, written by Matthew Wall at the Massachusetts Institute of Technology. The MPI version was written by Andrew Rasmussen \url{https://github.com/andyras/GAlib-mpi/blob/master/LICENSE} who modified the code from \url{https://github.com/B0RJA/GAlib-mpi}.
The evolver.cpp file is used to initialize the GA parameters and pipe the parameters to the GA Lib.
Detail of initial parameter settings of Genetic Algorithm for three different setup is given in the following table.
\begin{table}[h]
	\begin{tabular}{ |p{0.22\textwidth}|p{0.22\textwidth}|p{0.22\textwidth}|p{0.22\textwidth}| } 
		\hline
		\textbf{CPFA Parameters} & \textbf{Pheromone Only} & \textbf{Sitefidelity Only} & \textbf{All Parameters} \\
		\hline 
		Probability of Switching to Searching & U(0,1) & U(0,1) & U(0,1)\\ 
		\hline
		Probability of Returning to Nest & U(0,1) & U(0,1) & U(0,1)\\ 
		\hline
		Uniform Search Variation & (0, 4 PI) & (0, 4 PI) & (0, 4 PI)\\
		\hline
		Rate of Informed Searched Decay & E(20,0) & E(20,0) & E(20,0)\\
		\hline
		Rate of Site Fidelity & E(20,20) & E(20,0) & E(20,0)\\
		\hline
		Rate of Laying Pheromone & E(20,0) & E(20,20) & E(20,0)\\
		\hline
		Rate of Pheromone Decay & E(20,0) & E(20,20) & E(20,0)\\
		\hline
	\end{tabular}
	\caption{Initialization of seven parameters of CPFA for three different environments}
\end{table}
\section{\label{section:Generating Data Set for Analysis}Generating Data Set for Analysis}
 Once the parameters are tuned for three different environments, we have generated the data for our analysis using these parameter sets. For each of the experiment, we have extracted pick up and drop off time for each seed, location of each seed in the arena. Pick up time represents the time when the seed is picked by the ant from the location of the seed and drop time is when it is dropped off at the nest. We have tagged each ant with distinct ID. For each seed, we also have extracted which ant has collected that seed. Also, we have tracked when the pheromone is laid, and followed, and when the site fidelity is followed. We have assigned distinct ID number to each pile so that when a pheromone trail is laid we can track which pile the trail is coming from. For each type of environment, we have simulated $500$ experiments and generated data mentioned above. We varied the value of random seed for each experiment while keeping the CPFA parameters constant for a particular environment. We also varied the position of seeds for each experiment. Each experiment was performed for $90$ minutes. While generating the data for "Pheromone only parameters", We did not extract any site fidelity data, because we tuned all the parameters not to use site fidelity data. Similarly, for "site fidelity only" experiments we did not extract any pheromone data as there was no pheromone. We have collected both site fidelity data and pheromone data when we have used both methods together for collecting resources.
 \section{\label{Analyzing The Foraging Data}Analyzing The Foraging Data}
 After generating all the data from the simulation we have tried to observe how the ants collect seeds from different food distributions. We observed that it takes some time for them to discover the larger piles. But once they discover it, they start to collect seeds from those piles. And they use site fidelity and pheromone for this recruitment. Once they start collecting this seeds we see an increase in their foraging rate. So we tried to detect those changes in their foraging rate by applying the change point detection algorithm. 
 \subsection{\label{Creating Timeline for each type of distribution}Creating Timeline for each type of distribution}
 We have studied each experiment separately to analyze the change in their foraging rate. For each experiment, we studied foraging rate for each type of pile individually. To study foraging rate for each pile we have created a timeline for each type of distribution. 
 \begin{figure}[h]
 	\includegraphics[width=\textwidth]{TimeLine.jpg}
 	\caption{An example of a timeline for a distribution where numbers at the top represent the sliding window number. Values in the boxes are the rate of collection of seeds per window.}
 \end{figure}
 For example, we have divided the whole time of the experiments into sliding windows. Where each sliding window represents the rate of collection of seeds for that time. The length of the sliding window is $60$ seconds for \textit{P. Rugosus}. And then we slided the window by $10$ seconds so that we can get the rate of collecting seeds by that window. So if the experiment is for $90$ minutes ($5400$ seconds), we kept the length of the sliding windows for $60$ seconds and slided it by $10$ seconds, we get total $540$ sets of data where we calculated their foraging rate for a particular pile. Once we have created the timeline for each experiment for a particular distribution of seeds we used Change point detection algorithm to detect the change in the rate of collection of seeds. Another method we have created the timeline is by taking into consideration the change in the rate of foraging. In this method for creating the timeline instead of foraging rate, we take into consideration the change in foraging rate.\\ 
 The length of the window and the sliding amount for the window is different for each species. We have systematically varied this two parameters to fine tune the change point detection algorithms. Values for each of the parameters will be discussed further in the result section. 
 \begin{figure}[h]
 	\includegraphics[width=\textwidth]{ChangeInForagingRate.jpg}
 	\caption{An example of timeline and change in foraging rate. The change in foraging rate is calculated by measuring the difference between the timeline windows}
 \end{figure}
\section{\label{section:Change Point Detection Algorithm}Change Point Detection Algorithm}
 The change point detection algorithm is divided into two parts. First part is the adding rate of collecting seeds to calculate the cumulative sum and detrend for smoothing. And the second part is applying the change point detection algorithm. We have used binary segmented cumulative sum method to determine the change points.
 \subsection{\label{Calculating the Cumulative Sum}Calculating the Cumulative Sum}
 The calculation of cumulative sum is basically adding the foraging rate in each window. The following figure and algorithm show how the cumulative sum is calculated.
 \begin{figure}[h]
 	\includegraphics[width=\textwidth]{CumulativeSum.jpg}
 	\caption{This figure demonstrates how the cumulative sum is calculated from the timeline of foraging rate.}
 \end{figure}

 \begin{algorithm}[H]
 	\begin{algorithmic}[1]
 		\State Sum=0
 		\For{i=1:Number of Sliding Window} 
 			\State Sum= Sum + window(i)
 			\State CumulativeSum(i)=Sum
 		\EndFor
 		\caption{Pseudo code for calculating cumulative sum.}
 		\label{Pseudo code for calculating cumulative sum.}
 	\end{algorithmic}
 \end{algorithm}
 
 \subsection{\label{Detrending}Detrending}
 In time series trend means a lazy and gradual change in properties over the whole interval of the event. It is sometimes implicitly defined as a long-term change in the mean. It can also be referred as a change in some statistical properties. Usually, periodic and seasonal components and abrupt fluctuations and other parts were studied separately. Modern analysis techniques frequently treat the series without such routine decomposition, it is still required to consider the trend separately. The removal of a trend in a statistical or mathematical operation of time series is called detrending. It is often applied to remove features which are obsolete or unimportant. In time series analysis, detrending is also used in preprocessing step to prepare data set for further analysis. There are several methods of detrending. Linear trends in mean can be truncated by subtracting a least-square-fit straight line. Different procedures are used for more complicated trend. For example, the cubic smoothing spline is commonly used in dendrochronology to fit and remove ring-width trend that might not be linear, or not even monotonically increasing or decreasing over time. It is important to understand the effect of detrending on spectral properties of time series before trying to remove the trend from the time series. Before applying the change point detection algorithm, we have applied detrending algorithm to remove the trend from the time series. We used linear detrending and constant detrending to observe the effect of detrending in our time series data. Linear detrending removes the linear trend from the data where constant detrending removes the mean from the data. Here is an example, which shows how linear and constant detrending affects the time series.
 \begin{figure}[h]
 	\begin{subfigure}{\textwidth}
 		\centering
 		\includegraphics[width=\linewidth]{linearDetrending.jpg}
 		\caption{Linear Detrending}
 		\label{fig:Linear}
 	\end{subfigure}%
 \\
 	\begin{subfigure}{\textwidth}
 		\centering
 		\includegraphics[width=\linewidth]{constantDetrending.jpg}
 		\caption{Constant Detrending}
 		\label{fig:Constant}
 	\end{subfigure}
 	\caption{An example of applying linear and constant detrending on the cumulative sum of a timeline from one simulated CPFA experiment.}
 	\label{fig:fig}
 \end{figure}
\section{\label{section:Binary-Segmented Cumulative Sum}Binary-Segmented Cumulative Sum}
Binary Segmentation is one of the most established search method used for detecting the change point. This method extends any single change point method to multiple change points by iteratively repeating the method on different subsets of the sequence. 
To perform binary segmentation, we first apply the chosen single change point detection method to the entire data set, if no change point is found then we are done. If a change point is detected, call this τ, then the data is split into two segments, timeline$[1:t]$ and timeline$[t+1:n]$. We then apply the single change point method to the two segments and repeat iteratively. We stop when no more change points are detected. 
\\Binary segmentation is a very fast algorithm with complexity $O(n\log n)$ to detect the changes. But the major disadvantage of its computational correctness is that it gives us only an approximation of changes. It is not guaranteed that the binary segmentation method will find us the optimum solution. Also due to iterative nature of this algorithm, it may not detect changes in small changes. Which is why to verify the how well this method is performing, we have verified the results with the simulated data. 
The pseudo code for the binary segmentation algorithm is given in Algorithm $3.2$. 

\begin{algorithm}[!h]
	\begin{algorithmic}[1]
		 \State \algorithmicrequire A set of data of the form ($value_1$,$value_2$,$value_3$...)\\
		\qquad\quad\enspace A test statistic $\tau(.)$\\
		\qquad\quad\enspace An estimator of the changepoint position $\tau(.)$\\
		\qquad\quad\enspace A rejection threshold $\beta$
		\State\textbf{Initialize:} Let $C=\phi$, and $S=[1:n]$
		\While {$S\neq\phi$}
			\State Choose an element of S
			\State Denote this element as $[s,t]$
			\If{$ \tau (ys:t)<\beta $}
				\State remove$[s,t] from S$
			\EndIf
			\If {$\tau(ys:t)\geq\beta$}
				\State remove$[s,t] from S$
				\State calculate $ r=\tau(ys:t)+s-1 $, 
				\State add r to C
				\If {$r\neq s$}
					\State add $[s,r]$ to $S$
				\EndIf
				\If {$r=t-1$}
					\State $[r+1,t]$ to $S$
				\EndIf
			\EndIf
		\EndWhile	
		\caption{Pseudocode for Binary Segmented Mean Cumulative Sum}
		\label{Pseudocode for Binary Segmented Mean Cumulative Sum}
	\end{algorithmic}
\end{algorithm}

\section{\label{section:Verification of Change Points}Verification of Change Points}
As we have simulated data, and we know when the pheromones and site fidelity are used in simulations we can certainly verify how efficient is our change points detection algorithms are. So to check how efficient is our algorithms to detect change points, we divided the detection of change points into 4 categories.
\begin{itemize}
	\item \textbf{Catagory A:} Change point detection within 10 seconds of pheromone laying events, 
	\item \textbf{Catagory B:} Change point detections within 11-300 seconds of pheromone laying events, 
	\item \textbf{Catagory C:} Change point detections after more than 300 seconds of pheromone laying events and 
	\item \textbf{Catagory D:} Change point detected but no pheromone laying events has happened. 
\end{itemize} 
%So we have compared the results of four different methods based on this 4 categories using the data extracted from our simulation.
\section{\label{section:Applying the Best method on Field Data}Applying the Best method on Field Data}
We applied change point detection on 4 different types of the data set. This lead us to evaluate the performance of change point detection algorithm for four different methods.
\begin{figure}[h]
	\includegraphics[width=\textwidth]{ChangePoint.png}
	\caption{Four Different Change point Detection Method}
\end{figure}
After validating four different methods that we have applied to simulation data. We select the best method for them based on the performance on four categories stated above. And apply the best method in our field data.   


\chapter{Results}
\section{\label{section:Results from Simulation}Results from Simulation}
%\subsection{\label{Pheromone Only Parameters}Pheromone Only Parameters}
We configured the simulation settings as close as we can with the experimental setup of \textit{P. rugosus}, \textit{P. maricopa} and \textit{P. desertorum}. Then we have compared the result of change point analysis on \textit{foraging rate}, \textit{change in foraging rate}, \textit{constant detrended foraging rate }, \textit{constant detrended change in foraging rate}, \textit{linear detrended foraging rate} and \textit{linear detrended change in foraging rate}.\par
\begin{figure}[]
	\includegraphics[width=\textwidth,height=0.5\textheight]{PheromoneOnly/AllPlot.png}
	\caption{Comparison of change point detection method for pheromone only parameters. Outliers are skipped to provide a better indication of differences. The time in seconds represents the difference between a pheromone laying event followed by a detection of change point in the foraging rate.}
\end{figure} 
Figure $4.1$ compares the results of four different methods on a pheromone only simulated environment for \textit{P. rugosus}. We can see that, \textit{constant detrending} detects change points which is closer to the pheromone laying event. The result of \textit{constant detrending with change in foraging rate} is closer to the \textit{constant detrending}, but \textit{linear detrending} and \textit{linear detrending with change in rate} doesn't perform well in this environmental settings.\par 

The average difference of pheromone laying event followed by a change point detection for \textit{constant detrending} method is $47.5$, $60$ and $81.5$ for one pile, four large piles and sixteen piles. It is also observed that, the range of time difference between a pheromone laying event followed by a change point detection event is also significantly reduced.\par 
We observe similar phenomenon for \textit{P. maricopa} and \textit{P. desertorum}. The box plots of comparison for these two species are provided in the appendices. The outliers are removed from the figure to make the figure more informative. The detailed figure with the outliers are provided in the appendices.\par 
We compare the result with the change point detection in raw foraging rate. We observe that, the average time difference between change point detection and pheromone laying event is similar to \textit{constant detrending}. \par
\begin{figure}[]
	\includegraphics[width=\textwidth,height=0.5\textheight]{PheromoneOnly/RawFit/12Ant_PhOnly_CP.png}
	\caption{Enlarged view of efficiency of change point detection algorithm on raw foraging rate. The time in seconds represents the difference between a pheromone laying event followed by a detection of change point in the foraging rate.}
\end{figure}
We have also evaluated four different methods based on the four categories mentioned in section $3.7$.
As we can see from figure $4.3$ in \textit{constant detrending} , $15\%$ of the time change points are detected within 10 seconds of laying pheromone, whereas, in \textit{linear detrending} , it is only $9\%$. \textit{Constant detrending}  also has significant improvement over \textit{linear detrending}  in \textit{Catagory $11-300$}. $74\%$ of the time, changes points are detected with eleven to three hundred seconds of laying pheromone while using \textit{constant detrending} , on the other hand, using \textit{linear detrending} it is only $50\%$. In \textit{catagory $>300$} the error is $6\%$ for \textit{constant detrending} , whereas, it is $21\%$ for \textit{linear detrending}. \par
\begin{figure}[h]
	\includegraphics[height=0.35\textheight]{DonutCharts/Slide2.JPG}
	\caption{Efficiency chart for change point detection methods on a large pile of 256 seeds. This result is generated from the simulated data of pheromone only parameters, configured for \textit{P. rugosus}}
\end{figure}
\begin{figure}[H]
	\includegraphics[width=\linewidth, height=0.4\textheight]{DonutCharts/Slide3.JPG}
	\caption{Efficiency chart for change point detection methods on four medium  piles of 64 seeds. This result is generated from the simulated data of pheromone only parameters, configured for \textit{P. rugosus}}
\end{figure}
Performance of \textit{constant detrending method} and \textit{constant detrending method with difference in rate} is similar, if we consider \textit{categories $\le 10$} and \textit{catagory $11-300$}. But \textit{constant detrending} does significantly better than \textit{constant detrending with change of foraging rate} in detecting change points for four piles and sixteen piles in the simulated environment.\par
\begin{figure}[H]
	\includegraphics[width=\linewidth, height=0.4\textheight]{DonutCharts/Slide4.JPG}
	\caption{Efficiency chart for change point detection methods on sixteen small piles of 16 seeds. This result is generated from the simulated data of pheromone only parameters, configured for \textit{P. rugosus}}
\end{figure}
From figure $4.4$ and figure $4.5$ we can observe that, the performance of \textit{constant detrending} combined in \textit{category $\le 10$} and \textit{catagory $11-300$} is better than performance of \textit{constant detrending with change in foraging rate} combined in these two categories. Also from figure $4.4$ and $4.5$ we can see that \textit{constant detrending} does better in \textit{category $>300$} than \textit{constant detrending with change in foraging rate}. \par
If we look at the performance of the four methods over these categories in phermone only simulated environments, for \textit{P. maricopa} and \textit{P. desertorum} we observe similar pattern of performances. The results for \textit{P. maricopa} and \textit{P. desertorum} are provided in the appendices.\par  
\begin{figure}[]
	\includegraphics[width=\linewidth, height=0.4\textheight]{PheromoneOnly/RawFit/12AntsRawFitPHOnly.PNG}
	\caption{Efficiency chart for change point detection methods on foraging rate and change in foraging rate. This result is generated from the simulated data of pheromone only parameters, configured for \textit{P. rugosus}}
\end{figure}
Figure $4.6$ shows efficiency of change point detection algorithm on \textit{raw foraging rate} and \textit{change in foraging rate} in pheromone only experiments. We observe that, the performance is better for \textit{change in foraging rate}. \par
Figure $4.7$ shows the comparison of four different methods on a simulated environment with both memory and communication for \textit{P. rugosus}. From this figure it is clearly visible that \textit{constant detrending} does better in detecting the change points than any other methods. As a matter of fact, \textit{constant detrending} does better in memory plus communication environment than the communication only environment.\par 

\begin{figure}[H]
	\includegraphics[width=\textwidth,height=0.6\textheight]{AllParameters/AllPlot.png}
	\caption{Comparison of change point detection methods without outliers for \textit{pheromone plus sitefidelity} parameters . The time in seconds represents the difference between a pheromone laying event followed by a detection of change point in the foraging rate.}
\end{figure}
\begin{figure}
	\includegraphics[width=\textwidth,height=0.4\textheight]{AllParameters/RawFit/12Ants_SFandPH_WithRate_RoundTripWindow.png}
	\caption{Enlarged view of efficiency of change point detection algorithm on \textit{raw foraging rate} on simulation data of 12 ants. The time in seconds represents the difference between a pheromone laying event followed by a detection of change point in the foraging rate. Compared to figure 4.7 change point detection on raw data detects changes fastest.}
\end{figure}
\begin{figure}[h]
		\centering
		\includegraphics[width=\linewidth]{Comparison.PNG}
	\caption{Figure on the left is the efficiency chart of the change point detection algorithm on \textit{raw foraging rate with no detrending} for one large pile of 256 seeds. The data is generated from the simulation of pheromone plus site fidelity parameters of \textit{P. rugosus}. 
		Figure on the right is the efficiency chart of change point detection algorithm on \textit{raw foraging rate with no detrending} for one large pile of 256 seeds. The figure is generated by analyzing the simulated data of sitefidelity only parameters for \textit{P. rugosus}. }
	\label{fig:fig}
\end{figure}
\begin{figure}
	\includegraphics[width=\textwidth,height=0.4\textheight]{AllParameters/RawFit/12AntsRawFitSFAndPH.PNG}
	\caption{Efficiency of change point detection algorithm for detecting pheromone using\textit{ raw foraging rate} and \textit{change in raw foraging rate}, in memory and communication parameters.}
\end{figure}
\begin{figure}
	\includegraphics[width=\textwidth,height=0.4\textheight]{AllParameters/RawFit/SiteFidelityGraph.PNG}
	\caption{Efficiency of change point detection algorithm on raw data for detecting site fidelity in memory and communication parameters.}
\end{figure}
\begin{figure}[h]
	\includegraphics[height=0.4\textheight,width=\linewidth]{NewResult/RandomSeed_43109511.png}
	\caption{A plot of seed collection from simulation of 12 simulated ants. The stars in the foraging data represent the pheromone laying event and circles represent detection of change points using the binary segmented cumulative sum method on raw data.}
\end{figure}
We have analyzed the efficiencies of all four methods for communication plus memory environmental settings. We observe that \textit{constant detrending} and \textit{constant detrending with change in foraging rate} does significantly better than the other two methods. But in \textit{category $>300$} and catagory None, \textit{constant detrending} does better than \textit{constant detrending with change in rate}. We observe that the error is 0\% for \textit{constant detrending}, whereas it is 1\% for \textit{constant detrending with change in foraging rate}. \textit{Constant detrending} also surpasses \textit{constant detrending with change in foraging rate} in the combined result of \textit{category $\le 10$} and \textit{category $11-300$}. We observe, the similar pattern of performances for four piles and sixteen piles. \par
Figure $4.8$ shows the efficiency of change point detection algorithm on foraging rate. We observe, time to detect the change points is better than \textit{constant detrending}. \par%upto this
Figure \textit{4.9(a)} is the efficiency chart for \textit{foraging rate with no detrending} for one large pile on pheromone plus site fidelity environment. The result of other piles for \textit{P. rugosus} have similar pattern. We have mentioned those in appendices. In figure \textit{4.9(b)} we demonstrate our model\textquotesingle s efficiency in detecting the change points when only site fidelity is used for \textit{P. rugosus} on one large pile of 256 seeds.\par 
In memory only environment when no pheromone is used, agents use memory extensively. For this reason, when they start collecting seeds, their foraging rate goes up because of the extensive use of internal memory. Which is why \textit{constant detrending} also method detects change points in site fidelity only environment. So we have also measured the efficiency of all methods by measuring the difference between a use of site fidelity and followed by a change point. From figure 4.9(b), we have observed that 58\% of the time a change point is detected after the first use of site-fidelity.\par  
The performance rating is similar for \textit{P. maricopa} and \textit{P. desertorum}. Results for these two species are provided in the appendices. \par
To validate which method is better in detecting only pheromone but no site-fidelity event, we have also calculated the time difference between site fidelity and followed by a change point detection event for memory plus communication environment. From figure $4.10$ we observe that, only for few experiments it detects change points for the use of memory. This data is generated by applying the change point detection algorithm on \textit{foraging rate} and \textit{change in foraging rate}. We observe similar pattern for \textit{constant detrending} method. Although we miss some of the change points when we use the \textit{foraging rate with no detrending} or \textit{change in foraging rate with no detrending}, it is better because we do not detect much of the site fidelity events in this method.\par 
We have varied the number of ants to observe the change in the foraging rate in the simulation. We have performed the simulation with 96 ants and 48 ants. We have observed that in the simulation with 96 ants, they collect more random seeds than the seeds that are clumped. A reason for this is when the number of ants is too large, the rate of collision increases between them while they try to forage from a clumped food source. The performance is little better when we ran the experiment with 48 ants. They tend to collect seeds from the clumped food distribution. \par

We have applied the change point detection algorithm on the foraging rate of these two types of simulations, keeping the parameters for change point detection algorithm same. We observe that the average time difference between a pheromone laying event followed by a change point detection algorithm increases for 48 ants. And it is increased more for 96 ants. \par

We ran the simulation for 96 and 48 ants with site fidelity only parameters, we observe that they don't use the site fidelity often to recruit. Even though they use, it does not effect in their foraging rate. Although we have detected a few change points in their foraging rate, the accuracy of the algorithm is very low in the simulation for size fidelity only parameters. The results are included in the appendices. \par
The change point detection algorithm on raw foraging data detects the pheromone laying event fastest in pheromone only experiments. It also detects pheromone in memory and communication only environment. However, It rarely detects change points for sitefidelity in memory and sitefidelity experiments. It also detects change points in memory only experiments because sitefideity is used so often in these experiments. \par
Because it is the best at detecting changepoints, we use the binary segmented cumsum change point detection algorithm on raw data to detect recruitment in the field data. 
 
\section{\label{section:Results from Field Data}Results from Field Data} 
We applied the change point detection algorithm on the change of foraging rate in field data. The parameters are kept same as simulations for the change point detection method. We are able to detect change points in 10 experiments out of 11 for \textit{P. desertorum}. For \textit{P. maricopa}, we detected change points in 10 experiments out of 11 and for \textit{P. rugosus} we are able to detect change points in 11 experiments out of 13. Table $4.2$ shows the detail of the results for field data.\par
\begin{table}[H]
	\footnotesize
	\begin{tabular}{|c|c|c|c|}
		\hline
		\multicolumn{4}{|c|}{\textit{P. desertorum}} \\ \hline
		Experiment & One & Four & Sixteen \\ \hline
		DP15\_070709 & 4 & 2 & 2 \\ \hline
		DP10\_0606 & 3 & 3 & 2 \\ \hline
		DP14\_0606 & 2 & 2 & 2 \\ \hline
		DP15\_0604 & 2 & 0 & 2  \\ \hline   
		DP15\_0617 & 1 & 2 & 2 \\ \hline
		DPC\_070109 & 2 & 1 & 2  \\ \hline
		DP15\_070809 & 3 & 2 & 2  \\ \hline
		DP16\_0610 & 2 & 2 & 2  \\ \hline
		DPB\_070809 & 2 & 2 & 2  \\ \hline
		DP9\_0603 & 0 & 3 & 2  \\ \hline
	\end{tabular}
	\hfill
	\begin{tabular}{|c|c|c|c|}
		\hline
		\multicolumn{4}{|c|}{\textit{P. maricopa}} \\ \hline
		Experiment & One & Four & Sixteen \\ \hline
		MP8\_0609 & 3 & 3 & 4 \\ \hline
		MP9\_0528 & 2 & 2 & 1 \\ \hline
		MP9\_0602 & 1 & 2 & 2 \\ \hline
		MP9\_0611 & 2 & 2 & 1  \\ \hline   
		MP9\_0620 & 2 & 2 & 2 \\ \hline
		MPE\_24Jul09 & 2 & 2 & 2  \\ \hline
		MP4\_0527 & 2 & 2 & 2  \\ \hline
		MP7\_0620 & 1 & 2 & 2  \\ \hline
		MP8\_0527 & 1 & 1 & 1  \\ \hline      
		MPX\_0624 & 2 & 2 & 3  \\ \hline  
	\end{tabular}
	\hfill
	\\ \\
	\begin{center}
		\begin{tabular}{|c|c|c|c|}
			\hline
			\multicolumn{4}{|c|}{\textit{P. rugosus}} \\ \hline
			Experiment & One & Four & Sixteen \\ \hline
			RP6\_0609 & 2 & 2 & 2 \\ \hline
			RP7\_0610 & 3 & 2 & 1 \\ \hline
			RP10\_0527 & 2 & 2 & 2 \\ \hline
			RP10\_0528 & 1 & 2 & 2  \\ \hline   
			RP10\_0604 & 2 & 0 & 1 \\ \hline
			RP12\_0602 & 2 & 2 & 2  \\ \hline
			RP14\_0610 & 2 & 1 & 2  \\ \hline  
			RP16\_0612 & 2 & 2 & 2  \\ \hline
			RP19\_0618 & 3 & 2 & 2  \\ \hline
			RS23\_0821 & 2 & 2 & 2  \\ \hline 
			RP17\_0617 & 1 & 3 & 2  \\ \hline 
		\end{tabular}
	\end{center}
	\caption{This table represents the number of change points detected in each of the field experiment of \textit{P. desertorum}, \textit{P. maricopa} and \textit{P. rugosus}. For \textit{P. desertorum}, we are able to detect change points in 10 experiments out of 11. For \textit{P. maricopa} it is 10 out of 11. For \textit{P. rugosus} it is 11 out of 13. The numbers conclude how many times change points are detected for a pile type in each experiment.}
\end{table}
\begin{figure}[h]
 	\includegraphics[height=0.4\textheight,width=\linewidth]{NewResult/DP9.png}
 	\caption{A plot of seed collection from one of the field experiment of \textit{P. desertorum} with change points indicate with circles.}
\end{figure}
 
We ran 500 simulated experiment to generate the data for each species. And we are able to detect change points in all of the experiments. In the field experiments we were also able to detect change points for most of the experiments. 
This suggests all three species are either using pheromones for all pile sizes or they exclusively use sitefidelity, but use it in large amounts that generate change points.

\chapter{Conclusion}
Ants have been the center of attraction for research for so many myrmecologists. An enormous amount of research works has been performed to analyze the foraging strategies of ants. Scientists performed many biological experiments to discover what strategies ants follow to forage from the environment. \par
Although we have talked about the combination of three strategies for foraging it was unclear which of the strategies ants follow for recruiting from large food sources. Several indoor experiments on harvester ants which used high definition cameras and chemical detectors have shown that they lay pheromone trails for other ants to follow. \par
As the amount of pheromone laid by the ants are very small, it was difficult to identify the existence of chemicals in the outdoor field experiments. So, we used this mathematical approach to detect the pheromone laying events of ants in the outdoor field experiments. \par
In simulations, we could detect change points in every single experiment. But, we were not being able to detect change points in all the outdoor field experiments. It is because, in simulation, the CPFA is designed to forage, despite considering the condition of the environment. But as we have mentioned in the introduction about the factors which affects the foraging activity of harvester ants, can play a vital role in some of the experiments.\par
But since we can detect change points in the majority of the field experiments, we infer that ants use pheromone to recruit from large clumped food sources.
Also in ``memory only'' simulation environment ants used sitefidelity extensively, which resulted in the drastic change in their foraging rate. As a result, we can see picks and hikes in their foraging rate which was detected by our change point detection algorithm. But when we have tuned the simulations to use both memory and communication, we observe ants did not use memory much, even if they use, it didn’t affect their foraging rate, so a change point can be detected.\par
We have only used binary segmented cumulative sum methods for detecting the change points. In future, we can try to observe and find the pattern of changes in the foraging rate when the pheromone is laid, and use the statistical data from the simulation to infer the pheromone laying events in the field experiment data.

\chapter{Appendicies}
\section{\label{section:overview}Overview}
   The classic approach to proving a theorem is some really difficult 
   mathematics.  For the theory of relativity, I asked grandpa Al exactly 
   how he proved it.  He gave me a few hints, including some stuff about
   rest mass and big electro-motive force.  I think he is really smart.
\section{Conclusions}
   I conclude that this is a really short thesis.

\bibliographystyle{plain}
\bibliography{bibfile_name}

\end{document}
